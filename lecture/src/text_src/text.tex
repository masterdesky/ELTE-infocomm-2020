\section{Motivation}
Since the end of the Second World War, the U.S. Air Force (USAF) started to deeply analyse the circumstances of a destructive enemy attack and tried to come up with a viable plan for any possible scenario. The beginning of the Cold War further escalated this research. One of the topics included in these discussions -- arisen in the late $1950$s -- was the development of a withstanding communication network, which could function even after a targeted nuclear attack (CITE BARAN).

One of the USAF's large-scale projects codenamed as \textit{Project RAND} started in $1945$ and lead by the RAND Corporation since $1948$ was the one intended to connect military planning with research and development decisions in any field possible (CITE). The copious amount of topics the corporation worked on also encompassed the research of communication networks for military use and thus lead the development of a \q{survivable} network, sketched above.

Paul Baran started working on the project in $1959$ and published $11$ memorandum in $1964$, detailing the components and operation of the perfect network for the mentioned task.